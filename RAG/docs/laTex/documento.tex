\documentclass[12pt, a4paper]{article}
\usepackage[utf8]{inputenc}
\usepackage[T1]{fontenc}
\usepackage{amsmath}
\usepackage{graphicx}
\usepackage{amssymb}
\usepackage{listings}
\usepackage{hyperref}
\usepackage{enumitem}
\usepackage{geometry}
\geometry{a4paper, margin=1in}

\title{Implementación de un Sistema RAG para el Curso de Introducción a la Programación de la Universidad de Matanzas}
\author{Tu Nombre (o el nombre del equipo)}
\date{\today}

\begin{document}

\maketitle

\tableofcontents

\newpage

\section{Introducción}

    \subsection{Contexto y Motivación}
        \begin{itemize}
            \item Breve descripción del curso de Introducción a la Programación en la Universidad de Matanzas.
            \item Desafíos comunes en el aprendizaje de programación para estudiantes principiantes (dificultad en la búsqueda de información específica, necesidad de respuestas contextualizadas, etc.).
            \item Justificación de la implementación de un sistema RAG como solución para mejorar el aprendizaje y el soporte a los estudiantes.
        \end{itemize}

    \subsection{Objetivos del Proyecto}
        \begin{itemize}
            \item Objetivo general del proyecto: mejorar el acceso a la información y el soporte de aprendizaje mediante la implementación de un sistema RAG.
            \item Objetivos específicos:
                \begin{itemize}[label=\textbullet]
                    \item Facilitar el acceso a información relevante de los materiales del curso.
                    \item Proporcionar respuestas contextualizadas y personalizadas a las preguntas de los estudiantes.
                    \item Mejorar la comprensión de los conceptos de programación.
                    \item Aumentar la autonomía y el compromiso de los estudiantes con el curso.
                \end{itemize}
        \end{itemize}

    \subsection{Alcance del Proyecto}
        \begin{itemize}
            \item Definición de los materiales del curso que se utilizarán como base de conocimiento (conferencias en PDF, guías de estudio, ejercicios).
            \item Tipos de preguntas que el sistema RAG deberá poder responder (conceptos teóricos, sintaxis de código, resolución de problemas).
            \item Usuarios objetivo: estudiantes del curso de Introducción a la Programación.
        \end{itemize}

\section{Diseño del Sistema RAG}

    \subsection{Arquitectura General del Sistema}
        \begin{itemize}
            \item Descripción de la arquitectura del sistema RAG: componentes (modelo de recuperación, modelo de generación, base de conocimiento), flujo de información.
            \item Diagrama de la arquitectura del sistema.
            \begin{figure}[h]
                \centering
                \includegraphics[width=0.8\textwidth]{ruta_a_tu_diagrama.pdf}
                \caption{Diagrama de la arquitectura del sistema RAG}
            \end{figure}
        \end{itemize}

    \subsection{Preprocesamiento de los Documentos}
        \begin{itemize}
            \item Extracción de texto de los PDFs (bibliotecas utilizadas: PyPDF2, pdfplumber, Tesseract OCR).
            \item Limpieza y normalización del texto (eliminación de caracteres especiales, minúsculas, lematización, etc.).
            \item Fragmentación del texto en unidades semánticas (párrafos, secciones, etc.).
        \end{itemize}

    \subsection{Modelo de Recuperación}
        \begin{itemize}
            \item Selección del modelo de embeddings (Sentence Transformers, OpenAI Embeddings, otros).
            \item Justificación de la elección del modelo.
            \item Generación de embeddings para los fragmentos de texto.
            \item Creación del índice de búsqueda vectorial (Faiss, Annoy, bases de datos vectoriales).
            \item Optimización del índice.
        \end{itemize}

    \subsection{Modelo de Generación}
         \begin{itemize}
            \item Selección del LLM (OpenAI GPT-3.5/4, Llama, Mistral, otros).
            \item Justificación de la elección del LLM.
            \item Diseño del prompt para el LLM para la utilización de la información recuperada.
         \end{itemize}


    \subsection{Integración del Modelo de Recuperación y Generación}
        \begin{itemize}
            \item Descripción del proceso de recuperación de fragmentos relevantes.
            \item Descripción del proceso de aumento de la consulta con la información recuperada.
            \item Descripción del proceso de generación de la respuesta por parte del LLM.
        \end{itemize}

\section{Implementación del Sistema}

    \subsection{Herramientas y Tecnologías Utilizadas}
         \begin{itemize}
             \item Bibliotecas y frameworks de Python utilizados: (Transformers, Faiss/Annoy, LangChain/LlamaIndex).
             \item Bases de datos vectoriales utilizadas: (si aplica)
             \item Entorno de desarrollo: (IDE, sistema operativo).
         \end{itemize}

    \subsection{Desarrollo del Pipeline de RAG}
        \begin{itemize}
           \item Pasos detallados para la implementación del pipeline (carga de documentos, preprocesamiento, generación de embeddings, indexación, búsqueda, generación de respuesta).
           \item Fragmentos de código relevantes (ejemplos).
                \begin{lstlisting}[language=Python, caption=Ejemplo de preprocesamiento]
                    def preprocess_text(text):
                        text = text.lower()
                        # ... otros pasos ...
                        return text
                \end{lstlisting}
         \end{itemize}

   \subsection{Despliegue del Sistema (si aplica)}
        \begin{itemize}
            \item Descripción del entorno de despliegue (servidor, nube, etc.)
            \item Tecnologías y herramientas utilizadas para el despliegue (Docker, etc.).
        \end{itemize}

\section{Evaluación del Sistema}

    \subsection{Diseño de la Evaluación}
        \begin{itemize}
           \item Definición de las métricas de evaluación:
               \begin{itemize}[label=\textbullet]
                   \item Relevancia de los documentos recuperados.
                   \item Precisión y calidad de las respuestas generadas.
                   \item Fluidez y coherencia de las respuestas.
                   \item Cobertura de las respuestas.
                \end{itemize}
            \item Metodología de evaluación:
                 \begin{itemize}[label=\textbullet]
                    \item Creación de un conjunto de datos de Ground Truth.
                    \item Pruebas con usuarios reales.
                 \end{itemize}
        \end{itemize}

    \subsection{Resultados de la Evaluación}
        \begin{itemize}
            \item Presentación de los resultados de la evaluación usando las métricas definidas.
             \begin{itemize}[label=\textbullet]
                 \item Presentación de los resultados de la evaluación usando las métricas definidas.
                 \item Análisis de los puntos fuertes y débiles del sistema.
                 \item Identificación de áreas de mejora.
            \end{itemize}
        \end{itemize}

\section{Análisis y Discusión}

    \subsection{Beneficios del Sistema RAG}
        \begin{itemize}
            \item Mejora del acceso a la información y la comprensión de conceptos.
            \item Reducción del tiempo de búsqueda y solución de dudas.
            \item Aumento de la autonomía del estudiante.
            \item Potencial para personalizar el aprendizaje.
        \end{itemize}

    \subsection{Desafíos y Limitaciones}
         \begin{itemize}
            \item Dificultades técnicas encontradas durante el desarrollo.
            \item Limitaciones del sistema RAG (comprensión de preguntas ambiguas, posibles errores en las respuestas).
            \item Potenciales sesgos y limitaciones en el conjunto de datos y los modelos.
            \item Recomendaciones para superar estos desafíos.
        \end{itemize}

\section{Conclusiones y Trabajo Futuro}

   \subsection{Conclusiones}
        \begin{itemize}
            \item Resumen de los logros del proyecto.
            \item Respuesta a los objetivos planteados.
            \item Reflexión sobre el impacto del sistema en el proceso de aprendizaje.
        \end{itemize}

   \subsection{Trabajo Futuro}
        \begin{itemize}
            \item Propuestas para mejorar el sistema RAG (fine-tuning, nuevos modelos, mejorar la calidad de datos).
            \item Posibles extensiones del sistema (integración con otras herramientas, personalización más avanzada).
            \item Investigaciones futuras.
        \end{itemize}

\section{Anexos}
    \subsection{Código fuente (snippets relevantes)}

    \subsection{Tablas y gráficos con resultados de evaluación.}

    \subsection{Glosario de términos técnicos.}

    \subsection{Bibliografía y referencias.}

\end{document}